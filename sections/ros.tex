\chapter{Robot Operating System}
\label{chapter:ros}

\textbf{Author: Lukas Leskovar} 

This chapter's objective is to describe the basic concepts of the Robot Operating System (ROS) utilised by Autumn. The ROS, despite its name, is a meta-operating system or middleware providing the utility and services often found in robotics frameworks. It enables the composition of distributed systems by utilising publisher-subscriber communication between different programs of such systems. Furthermore, ROS provides a comprehensive set of tools enabling the compilation, operation, testing, visualisation, and debugging of robotic systems. ROS facilitates the development of robotic applications without reimplementing standardised technology with its vast amount of libraries and a huge open-source community providing useful functionality. \footcite{openSourceRoboticsFoundationDefinitionNodate}


\section{Conceptual Overview}
The Robot Operating System can be divided into three conceptual levels, each contributing an integral part to the utility of ROS. These different levels are described in the following sections. \footcite{openSourceRoboticsFoundationConceptsNodate}

\subsubsection{File System}
The File System Level mainly provides constraints and best practices for creating and structuring packages and their components. 
ROS provides appropriate tools to facilitate file-system operations with and within packages.
%operations such as searching for packages or file within packages
%With appropriate tooling the ROS facilitates file-system operations within packages, messages or topics. 

\subsubsection{Computational Graph}
The Computational Graph provides crucial functionality to ROS as it refers to the peer-to-peer mesh network of processes (nodes), each providing data to be utilised within the graph by publishing and subscribing to topics.
The concepts and technologies powering the computational graph are described later in this chapter.

\subsubsection{Community}
The community preserves the usability of ROS as new and useful packages, and tools are created as well as existing functionality is being maintained.

\section{Naming}
To aid the organisation of programs, processes, and resources, ROS provides two naming schemes described in the following sections. \citereset\footcite{openSourceRoboticsFoundationConceptsNodate}

\subsubsection{Graph Resource Names}
Graph Resource Names utilise a hierarchical structure to organise nodes, services, topics or anything else within the computational graph. ROS defines four different types of names:
%\begin{itemize}
%	\item base names - Are resolved in the same fashion as relative names
%	\item relative names - Are resolved relatively starting by the nodes name
%	\item global names - Begin with a / and are considered fully resolved
%	\item private names - Begin with a $\tilde{ }$ and convert the nodes name into a namespace
%\end{itemize}

\subsubsection{Package Resource Names}
Package Resource Names aim to facilitate the search process of resources at File System Level. These names usually consist of the package's name and the path to the desired resource within the package. 
%Examples for such Names are:
%\begin{itemize}
%	\item 
%\end{itemize}

\section{Packages}
Software in ROS is organised in packages containing nodes, libraries or any other piece of software providing functionality.

%noch ein bisschen trennen (atomic build item, ...)
%der absatz gefällt mir nicht, eigentlich gehört da nur die atomicity hin
%In order to maintain reusability, atomicity and easy decoupling of functionality packages aim to be as slim as possible by implementing only a limited-set of features. This means that each package is develop to focus on one task alone and work together with other packages to deliver utility as a connected system.
Since packages are the atomic unit of build and release, they aim to be as slim as possible by implementing only a limited set of features. 
In other words, packages should be implemented to provide minimal usability without being too large-scaled.\footcite{openSourceRoboticsFoundationPackageNodate} This means that each package is developed to work together with other packages to deliver utility as a connected system.  

At file-system level, packages refer to directories. While most subfolders and files within a package depend on their purpose, every package has to contain a package.xml and a CMakeLists.txt providing meta and build information.
Packages can be build by utilizing rosbuild or catkin. \footcite{openSourceRoboticsFoundationBuildNodate}

\subsubsection{Metapackages} 
Metapackages are specialized packages only containing a package.xml that logically links multiple related packages.\footcite{openSourceRoboticsFoundationMetapackageNodate}
They can be used to conveniently install a group of packages simultaneously. %naja ned so wirklich



\section{Nodes}
The goal of ROS is to promote code reusability and decoupling of functionality to aid the versatility and usability of the system. 
Following this guideline, every robotic system utilising ROS consists of a fine-grained graph of processes called nodes. Each node provides computation on a single feature utilising a ROS client library to communicate with others over a mesh-like peer-to-peer network. \footcite{openSourceRoboticsFoundationNodesNodate}

Exemplary for such a system would be one node running a LiDAR sensor, one responsible for localisation, one performing motion planning, one controlling motor drivers and motors, and one node running the robot's main control loop.

This architecture allows for much more fault safe and less complex applications in comparison to monolithic systems. \footcite[Page 94]{stephensBeginning2015}
This means that development and debugging are facilitated since errors can be contained within a singular slim node rather than a more extensive program. 

Each node has a node type consisting of the package name it is located and the nodes executable. 



\section{Communication}

\subsection{Messages}
Messages are the medium of communication used in topics or services to transport data between nodes. 
%They are used to send data between nodes over topics or through services. 

\subsubsection{Message Description}
A message is a simple data structure consisting of multiple type fields. These fields can be primitives, arrays, custom types as well as other message types. \footcite{openSourceRoboticsFoundationMessagesNodate}

The message description language can be used to structure custom messages in
\textit{.msg} files contained in the \textit{msg} directory of a package.

\subsubsection{Message Types}
Message types refer to package resource names consisting of the package's name as well as the name of the messages \textit{.msg} file.


\subsection{Topics}
The core component of communication in ROS are topics. They are unidirectional message streams enabling data transmission by utilising the publisher-subscriber model 
% utilising the publisher-subscriber model to establish data transmission in a many-to-many relationship.
Furthermore, the decoupling of functionality is facilitated by anonymously connecting nodes as producers and consumers of data. This means neither publishers nor subscribers of the topic need to know each other. 
While ROS does not limit the number of publishers and subscribers connected to a topic, it strictly enforces the usage of the exact message type specified for the topic's communication to work properly.

\subsection{Services}
The communication architecture in ROS utilising the publisher-subscriber model is advantageous in most use-cases, however most distributed systems require remote procedure calls (RPC) which are not supported by default.

With RPCs, a client sends a request to a server specifying the procedure to be called and its parameters. While the server executes the procedure, the client awaits a reply. Once the procedure's results are computed and sent to the client, its workflow can be resumed.\footcite[Page 3]{rfc1831}

Services enable communication over RPC by defining a pair of messages, one for requests and one for replies. Such service can then be attached to a node and called by a client using the service name. \footcite{openSourceRoboticsFoundationServicesNodate}




\section{Master}
One of the most important components of ROS is the Master. It tracks publishers and subscribers of topics and services and provides registration and name resolution to nodes. This means whenever a node wants to publish or subscribe to a specific topic or service, it contacts the Master first using XML-RPC. When a topic has at least one subscriber and publisher, the Master negotiates between the nodes to establish a peer-to-peer connection using a Slave API provided by the nodes XML-RPC Server.\footcite{openSourceRoboticsFoundationMasterNodate} A simplified version of this procedure can be seen in Figure \ref{fig:ros_master_reg}

Besides registration and name resolution, the ROS Master also provides a Parameter Server used for globally storing static system parameters.\footcite{openSourceRoboticsFoundationParameterServerNodate}
\begin{figure}[]
	\centering
	\includesvg[width=0.8\linewidth]{img/svg/ros_master_registration}
	\caption{Diagram of the topic registration process in ROS at which the publishing node registers its topic before the subscriber tells the master its interest in the point-cloud topic. However a node can be registered as a subscriber to a specific topic without the topic existing yet.}
	\label{fig:ros_master_reg}
\end{figure}



\section{Transform Library}
A complex robotic system consists of multiple parts such as sensors, cameras and manipulators, each represented as a coordinate frame, where each frame is connected to another frame using joints. When trying to move a specific part or coordinate frame, the transform of that single frame and the composite transform of each frame in relation to the target have to be calculated. This is especially important when moving a robotic arm based on sensor readings. In this example a transform between the position of the sensor and the arm needs to be calculated so the motion performed by the arm the motion perceived by the sensor match.
These complex calculations can be facilitated using the ROS Transform Library (tf). To this end, tf keeps track of each coordinate frame in an acyclic relationship tree where tf broadcasters then publish relative pose information, and listeners query transforms between two coordinate frames. 
Because not all pose information in a robotic system is instantly accessible, the tf saves this information for each frame over time. This means that transforms can be queried not just spatially but also temporally.



\section{Simulation}
Debugging and testing robot applications can be a repetitive and tedious task, especially when a test environment needs to be reset at every test cycle. In order to facilitate this part of development, future implementations of the Autumn Drone ought to utilize the representation and simulation technologies described in the following sections.

\subsection{URDF}
In order to perform simulations or compute coordinate frame transforms, a robot needs to be described in some way. One of the more popular description formats is the Unified Robot Description Format (URDF) which provides an XML format for representing a robot and its components, as well as a \textit{C++} parser and tools to convert and verify and visualise these models. \footcite{openSourceRoboticsFoundationURDFNodate}
The \textit{check\_urdf} tool parses a URDF-File and returns the robot's kinematic chain if successful.
To visualize the robots frames and joints in a graphviz\footcite{graphvizAuthorsAboutNodate} tree the \textit{urdf\_to\_graphiz} tool can be used. The resulting tree corresponds to the relationship tree tf uses to calculate transforms.

\subsection{Gazebo Simulator}\label{section:gazebo}
Gazebo is a 3D physics simulator often used in close relation to ROS projects. Therefore it provides tooling for model and world design and generation and comprehensive interfaces for controlling a simulated robot through ROS. Further advantages of Gazebo are its accurate sensor and sensor noise generation and its large community providing countless models of robots and sensors. \footcite{openSourceRoboticsFoundationGazeboNodate}
%vielleicht noch über URDF SDF schreiben - aber erst wenn simulation von autumn fertig ist

\filbreak