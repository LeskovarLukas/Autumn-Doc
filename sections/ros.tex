\chapter{Robot Operating System}

\textbf{Author: Lukas Leskovar} 

This chapter aims to describe the basic concepts of the Robot Operating System (ROS) utilized by the Autumn-Drone. The ROS despite its name is a meta-operating system or middleware providing the utility and services often found in other robotics frameworks. It enables the composition of distributed system by utilizing publisher-subscriber communication between different programs of such systems. Furthermore ROS provides a comprehensive set of tools enabling the construction, operation as well as testing, visualization and debugging of robotic systems. With its vast amount of libraries and huge open-source community providing useful functionality ROS facilitates the development of robotic applications without reinventing the wheel.


\section{Conceptual Overview}

\subsection{File System}
%With appropriate tooling the ROS facilitates file-system operations within packages, messages or topics. 

\subsection{Computational Graph}

\subsection{Community}



\section{Packages}



\section{Nodes}



\section{Services}



\section{Communication}

\subsection{Topics}

\subsection{Messages}



\section{Master}



\section{Transformation Library}



\section{Simulation}



\subsection{URDF}



\subsection{Gazebo Simulator}



\filbreak