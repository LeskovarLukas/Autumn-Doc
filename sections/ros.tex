\chapter{Robot Operating System}

\textbf{Author: Lukas Leskovar} 

This chapters objective is to describe the basic concepts of the Robot Operating System (ROS) utilized by Autumn. The ROS despite its name is a meta-operating system or middleware providing the utility and services often found in robotics frameworks. It enables the composition of distributed systems by utilizing publisher-subscriber communication between different programs of such systems. Furthermore ROS provides a comprehensive set of tools enabling the compilation, operation as well as testing, visualization and debugging of robotic systems. With its vast amount of libraries and huge open-source community providing useful functionality ROS facilitates the development of robotic applications without having to reimplement standardized technology. \footcite{openSourceRoboticsFoundationDefinitionNodate}


\section{Conceptual Overview}
The Robot Operating System can be divided into three conceptual levels each contributing a integral part to the utility of ROS. These different levels are described in the following sections. 

\subsubsection{File System}
The File System Level mainly provides constraints and best practices for creating and structuring packages and their components. \footcite{openSourceRoboticsFoundationConceptsNodate}
ROS provides appropriate tools to facilitate file-system operations with and within packages.
%operations such as searching for packages or file within packages
%With appropriate tooling the ROS facilitates file-system operations within packages, messages or topics. 

\subsubsection{Computational Graph}
The Computational Graph provides crucial functionality to ROS as it refers to the peer-to-peer mesh network of processes (Nodes) each providing data to be utilized within the graph by publishing and subscribing to topics.\citereset\footcite{openSourceRoboticsFoundationConceptsNodate}
The concepts and technologies powering the computational graph are described in later in this chapter.

\subsubsection{Community}
The Community preserves the usability of ROS as new and useful packages and tools are created as well as existing functionality is being maintained.


\section{Packages}
Software in ROS is organized in packages containing nodes, libraries or any other piece of software providing functionality.

%noch ein bisschen trennen (atomic build item, ...)
%der absatz gefällt mir nicht, eigentlich gehört da nur die atomicity hin
%In order to maintain reusability, atomicity and easy decoupling of functionality packages aim to be as slim as possible by implementing only a limited-set of features. This means that each package is develop to focus on one task alone and work together with other packages to deliver utility as a connected system.
Since packages are the atomic unit of build and release they aim to be a slim as possible my implementing only a limited set of features. 
In other words packages should be implemented to provide minimal usability without being too large-scaled.\footcite{openSourceRoboticsFoundationPackageNodate} This means that each package is developed to work together with other packages to deliver utility as a connected system.  

At file-system level packages simply refer to directories. While most subfolders and files within a package depend on its purpose, every package has to contain a package.xml and a CMakeLists.txt providing meta and build information.
Packages can be build by utilizing rosbuild or catkin. \footcite{openSourceRoboticsFoundationBuildNodate}

\subsubsection{Metapackages} 
Metapackages are specialized packages only containing a package.xml that logically links multiple related packages.\footcite{openSourceRoboticsFoundationMetapackageNodate}
They can be used to conveniently install a group of packages simultaneously. %naja ned so wirklich



\section{Nodes}
The goal of ROS is to promote code reusability and decoupling of functionality to aid the versatility and usability of the system. 
Following this guideline every robotic system utilizing ROS consists of a fine-grained graph of processes called nodes. Each node provides computation on a single feature utilizing a ROS client library to communicate with others over a mesh-like peer-to-peer network. \footcite{openSourceRoboticsFoundationNodesNodate}

Exemplary for such as system would be one node running a LiDAR sensor, one responsible for localization, one performing motion planning, one controlling motor drivers and motors as well as one node running the robots main control loop.

This architecture allows for a much more fault safe and less complex applications in comparison to monolithic systems. \footcite[Page 94]{stephensBeginning2015}
This means that development and debugging are facilitated since errors can be contained within a singular slim node rather than a larger program. 

To facilitate the search process each node has a node type consisting of the package name it is located and as well as the executable needed to run such node. 

\subsubsection{Services}
The communication architecture in ROS utilizing a publisher-subscriber model is advantageous in most use-cases, however most distributed systems require remote procedure calls (RPC) which are not supported by default.
%noch einen satz über rpc
Services enable communication over RPC by defining a pair of messages, one for requests and one for replies. Such service can then be attached to a node and called by a client using the service name. \footcite{openSourceRoboticsFoundationServicesNodate}


\section{Communication}

\subsection{Topics}

\subsection{Messages}



\section{Master}



\section{Transformation Library}



\section{Simulation}



\subsection{URDF}



\subsection{Gazebo Simulator}



\filbreak