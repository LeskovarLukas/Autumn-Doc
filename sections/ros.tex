\chapter{Robot Operating System}

\textbf{Author: Lukas Leskovar} 

This chapters objective is to describe the basic concepts of the Robot Operating System (ROS) utilized by Autumn. The ROS despite its name is a meta-operating system or middleware providing the utility and services often found in robotics frameworks. It enables the composition of distributed systems by utilizing publisher-subscriber communication between different programs of such systems. Furthermore ROS provides a comprehensive set of tools enabling the compilation, operation as well as testing, visualization and debugging of robotic systems. With its vast amount of libraries and huge open-source community providing useful functionality ROS facilitates the development of robotic applications without having to reimplement standardized technology. \footcite{openSourceRoboticsFoundationDefinitionNodate}


\section{Conceptual Overview}
The Robot Operating System can be divided into three conceptual levels each contributing a integral part to the utility of ROS. These different levels are described in the following sections. 

\subsubsection{File System}
The File System Level mainly provides constraints and best practices for creating and structuring packages and their components. \footcite{openSourceRoboticsFoundationConceptsNodate}
ROS provides appropriate tools to facilitate file-system operations with and within packages.
%operations such as searching for packages or file within packages
%With appropriate tooling the ROS facilitates file-system operations within packages, messages or topics. 

\subsubsection{Computational Graph}
The Computational Graph provides crucial functionality to ROS as it refers to the peer-to-peer mesh network of processes (Nodes) each providing data to be utilized within the graph by publishing and subscribing to topics.\citereset\footcite{openSourceRoboticsFoundationConceptsNodate}
The concepts and technologies powering the computational graph are described in later in this chapter.

\subsubsection{Community}
The Community preserves the usability of ROS as new and useful packages and tools are created as well as existing functionality is being maintained.


\section{Packages}
Software in ROS is organized in packages containing nodes, libraries or any other piece of software providing functionality.\footcite{openSourceRoboticsFoundationPackageNodate} 

In order to maintain reusability, atomicity and easy decoupling of functionality packages aim to be as slim as possible by implementing only a limited-set of features. This means that each package is develop to focus on one task alone and work together with other packages to deliver utility as a connected system.

At file-system level packages simply refer to directories. It has to contain a package.xml and a CMakeLists.txt file describing build and meta-information about the package and its components.
Packages can be build by utilizing rosbuild or catkin. \footcite{openSourceRoboticsFoundationBuildNodate}

\subsubsection{Metapackages} 
Metapackages are specialized packages only containing a package.xml that logically links multiple related packages.\footcite{openSourceRoboticsFoundationMetapackageNodate}
They can be used to conveniently install a group of packages simultaneously. %naja ned so wirklich



\section{Nodes}

\subsubsection{Services}



\section{Communication}

\subsection{Topics}

\subsection{Messages}



\section{Master}



\section{Transformation Library}



\section{Simulation}



\subsection{URDF}



\subsection{Gazebo Simulator}



\filbreak