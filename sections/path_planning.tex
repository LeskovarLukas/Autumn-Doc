\chapter{Path Planning}

\textbf{Author: Fabian Kleinrad} 

A crucial part of autonomy in robotics are means for planning ahead movements in a cooperative manner with the environment. Means to accomplish this are so called path planning algorithms. This chapter is going to focus on exploring the different kind of approaches to path planning and evaluate which approach is most fitting to be used in a real-time, high-dimensional use case present in the Autumn project.


\section{Algorithm Variants}

The problem of finding an optimal path between two points is an old one.
The first proposed solution was the Dijkstra's Algorithm. However with steadily evolving computer science the challenges to be master by such Algorithms got harder and harder. That's the reason why over the last years the simple principle of the Dijkstra Algorithm has branched out specializing and excelling in certain real world applications.
\footcite{Pan2020}

\subsection{Sampling-based Algoritms}

In motion planning Sampling-based Algorithms can be differentiated to other kinds of approaches, by the way they explore their environment. Sampling-based Algorithms such as the Probabilistic Road Map Algorithm or the Rapidly exploring Random Tree use a random point in their reference space and expand in that direction. This random point is considered a sample.

\subsection{Multiple-Query and Single-Query}

The term Multiple-Query refers, in connection with path planning Algorithms, to the feasibility of deriving variety of different paths, without the need of rerunning the algorithm. In Contrast Single-Query Algorithms are only able to compute one path at a time.\newline
Use cases for Multiple-Query Algorithms would be unchanging environments. The reason for that, by generating an extensive grid of connections to be able to calculate a multitude of different start/goal combinations  more computational time is needed.\newline
Single-Query approaches focus on performance instead of reuse-ability, which makes them ideal for dynamic domains. 
\footcite{Bekris2003}
\footcite{MultiSingleQuery}

\section{PRM}

\subsection{Core Principles}

\subsection{How it works}

\section{RRT Algorithm}

\subsection{Core Principles}

\subsection{How it works}

\section{A* Algorithm}

\subsection{Core Principles}

\subsection{How it works}

\section{Comparison RRT with A*}

\subsection{Purpose of RRT}

\subsection{Purpose of A*}

\subsection{Better choice for UAV application}

\section{RRT* Algorithm}

\subsection{Difference to RRT}

\subsection{How it works}

\section{Other RRT Variants}

\subsection{RT-RRT}

\subsection{Smart-RRT}
