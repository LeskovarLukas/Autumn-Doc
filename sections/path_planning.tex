\chapter{Path Planning}

\textbf{Author: Fabian Kleinrad} 

A crucial part of autonomy in robotics are means for planning ahead movements in a cooperative manner with the environment. Means to accomplish this are so called path planning algorithms. This chapter is going to focus on exploring the different kind of approaches to path planning and evaluate which approach is most fitting to be used in a real-time, high-dimensional use case present in the Autumn project.


\section{Algorithm Variants}

\subsection{Simple Pathfinding Algoritms in high Dimensions}

\subsection{Sampling-based Algoritms}

\subsection{Multiple Query}

\subsection{Single Query}

\section{PRM}

\subsection{Core Principles}

\subsection{How it works}

\section{RRT Algorithm}

\subsection{Core Principles}

\subsection{How it works}

\section{A* Algorithm}

\subsection{Core Principles}

\subsection{How it works}

\section{Comparison RRT with A*}

\subsection{Purpose of RRT}

\subsection{Purpose of A*}

\subsection{Better choice for UAV application}

\section{RRT* Algorithm}

\subsection{Difference to RRT}

\subsection{How it works}

\section{Other RRT Variants}

\subsection{RT-RRT}

\subsection{Smart-RRT}
