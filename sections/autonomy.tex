\chapter{Autonomous Navigation}

\textbf{Author: Lukas Leskovar} 
Autonomy or the capability of a robot to perform certain tasks without human supervision or intervention is a central topic in any robotic applications.
While stationary systems performing repetitive tasks can be automated relatively effortlessly, mobile robots introduce many challenges that need to be overcome in order to achieve autonomy.
This chapter deals with the classification and explanation of such robots to illustrate the applicability of Autumn as a autonomous navigation system.

\section{Autonomous Navigation}
To navigate a robot trough its environment autonomously it is necessary to develop an algorithm that employs the robot to move without any external controls. Regardless of how this algorithm is implemented any autonomous mobile vehicle consists of two fundamental abilities. The ability to perceive its surrounding using one or many sensors and to relocate itself using any equipped means of locomotion. 

\subsection{Reactive Approach}
One way to autonomously control a robot is to define simple behavioural rules that directly act upon sensory input without needing a sophisticated model of its environment. 
This approach can be associated with bionic robotics as its concepts are inspired by relatively simple life forms performing intelligent behaviour without having a brain. 
These behaviours match sensory input to movement and can be categorized into three levels of complexity:
\begin{itemize}
	\item Reflexes - pre programmed direct connections between stimuli and actions
	\item Reactions - learned behaviours that execute without the need of complex logic 
	\item Consciousness - sequences of reactive behaviours ruled by a logic architecture
\end{itemize}

Within a reactive system different behaviours are stacked without any knowledge of one another. This allows for layers to build upon functionality implemented by lower layers while maintaining decoupling of modules thus promoting task dissection and facilitate independent testing. 
%Robotic Paradigms and Control Architectures Jan Faigl

\subsection{Hierarchical Approach}
Contrary to reactive systems robots representing the hierarchical approach require a much more complex implementation as world modelling and sophisticated reasoning is needed. 
Any such system can be divided into three components that are executed sequentially:
\begin{itemize}
	\item Sense - the robot perceives its environment and creates an abstracted model of it (e.g. create a occupancy grid using a LiDAR sensor)
	\item Plan - using a model of the environment (e.g. find the shortest path to a given waypoint)
	\item Act - transform the plan into motion by controlling the robots manipulators
\end{itemize}

While this approach can introduce many difficulties concerning real-world representation or computational complexity this approach facilitates development of intelligent semi- or fully-autonomous vehicles. 
%Robotic Paradigms and Control Architectures Jan Faigl

\subsection{Hybrid Approach}
One way to combine the reactive paradigms simplicity with the intricate prevision of tasks as seen in hierarchical systems can be implemented using a hybrid approach. 
This concept uses deliberate planning algorithms to determine which reactive behaviour should be executed using a global world model, all while monitoring the success of each behaviour to determine if its beneficial to achieve a set goal (e.g find out if the robot moves to a waypoint or is stuck).
%Robotic Paradigms and Control Architectures Jan Faigl

\section{Difficulties}
While in theory many approaches can be applied to a 



\section{Degrees of Autonomy}

\section{Autonomous Navigation in Autumn}

%Controll Cylcle (Sense, Perceive, Plan, Act)

\filbreak