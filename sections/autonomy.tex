\chapter{Autonomous Navigation}

\textbf{Author: Lukas Leskovar} 
Autonomy or the capability of a robot to perform certain tasks without human supervision or intervention is a central topic in any robotic applications.
While stationary systems performing repetitive tasks can be automated relatively effortlessly, mobile robots introduce many challenges that need to be overcome in order to achieve autonomy.
This chapter deals with the classification and explanation of such robots to illustrate the applicability of Autumn as a autonomous navigation system.

\section{Autonomous Navigation}
To navigate a robot trough its environment autonomously it is necessary to develop a algorithm that employs the robot to move without any external controls. Regardless of how this algorithm is implemented any autonomous mobile robot consists of two fundamental abilities. The ability to perceive its surrounding using one or many  sensors and to relocate itself using any equipped means of locomotion. 

\section{Reactive Approach}

\section{Optimal Approach}

\subsection{Models}

\section{Taxonomy}

\subsection{Dynamic Environments}

\subsection{Predictability}

\section{Degrees of Autonomy}

\section{Autonomous Navigation in Autumn}

%Controll Cylcle (Sense, Perceive, Plan, Act)

\filbreak