\chapter{Conclusion}

\textbf{Author: Fabian Kleinrad} 

\section{Autumn Result}

\subsection{Goal}
The objective of the autumn project being a fully autonomous drone, in order to create 3D scans of hard to maneuver environments. Additionally on basis of the technologies used, it is possible to deploy the drone in areas lacking a GPS signal. Technologies realizing these features are already available. In contrast to these commercially available solutions, the autumn project tries to accomplish these feats using a low budget approach. 

\subsection{Design choices}
This leads to the final design of the autumn drone.\newline
Crucial to the whole operation are means to perceive object surrounding the drone. To accomplish this, a stereo camera was chosen due to low cost, compared to mapping devices used for commercial autonomous solutions.\newline
The drone used in the autumn project provides for a wide range of possible sensor configuration and a high payload capacity. This simplified prototyping and helped with the fast iterative approach used in this project. Using a smaller drone would have impeded tests, because of the increased complexity of working with smaller physical space and weight margins.\newline
To enable computation on the drone itself, a NVIDIA-Jetson TX2 is being employed. This solves the issue of latency problems, which are especially troublesome in processes that need to perform without interruption to guarantee the most optimal results. In the case of the autumn project, such a process would be the continuous mapping of the environment.

\pagebreak
\subsection{Result}
The autumn drone can perform mapping using stereo images provided by the stereo camera mounted on the drone. These mapping results are then used by an RRT* based path-planning algorithm, in order to calculate a path originating from the position of the drone, to an selected end point. The current position is also provided by the SLAM algorithm in the mapping stage. Using these route information the drone can be controlled accordingly.
In the current version of the drone only semi-autonomous flight is supported. Due to the lack of testing possibilities and the risk that accompanies testing autonomous drones, flight capabilities were only assessed using user input.

\subsection{Mapping}
With the focus of autumn centering around creating 3D scans of environments, the aspect of mapping is a crucial factor, that the quality of the results depends on.\newline
Mapping was realized by implementing an SLAM algorithm. The rational of that decision is due to the fact that SLAM uses stereo images in order to perform it's localization and mapping.
The algorithm determines the current position of the device that provides the images. Based on this relative position, SLAM generates a point cloud representing the environment the drone explores. All mapping logic is computed on the drone using a NVIDIA-Jetson TX2, which provides enough computational power to ensure the most optimal results. In order to transfer the point cloud data, an access point is being hosted, over which data is transferred. This enables the user to get a real-time view of the model and how it is being constructed. The resulting point-cloud can then be used as reference material.\newline
An example for it's application would be the basis of an detailed render. In this scenario it would simplify the process of measuring the environment an transferring these into a modeling software. Using a point cloud, the defining structure is already present and only few adjustments have to be made.

\subsection{Path-Planning}
Autonomous implies to posses the means to move through an environment without an external help. In autumn this is realized using an path-planning algorithm. This algorithm is based on the RRT* algorithm, often employed in high-dimensional and dynamic scenarios. The algorithm plans a path through an either two-dimensional or three-dimensional representation of the environment surrounding the drone. This model is being generated in the mapping phase using the SLAM algorithm.\newline
The path is computed separate from the drone. The reason being, by separating the path computation onto an external device it enables the drone to work more efficiently and the path-planning algorithm to not be constricted by computational restriction present, when computing on the drone itself. The communication is realized through a wifi connection.\newline
The route is calculated between the current position of the drone and an user defined end-point. In the event of newer mapping data being available, the previously generated path gets checked for possible collision with newly scanned obstacles. Due to the single-query approach of the RRT algorithm a new path has to be calculated in the case of an invalid path.´
The resulting path can be visualized for the user alongside the model generated in the mapping phase. This allows for early error detection through an human observer.

\section{Outlook}

\subsection{Current problems}
At current times fully autonomous flight is not possible with the autumn drone. A major risk factor in the current design, being the inability to monitor the space above the drone.  This results in problems when using three dimensional path-planning because of the uncertainty of what lies above. Without that information the drone is unable to utilize the third-dimension, that is the most important factor when using an UAV. In autumn this was solved by focusing on an semi-autonomous approach.  
Another problem is the short battery life-time of the drone, which is due to the size of the drone used in the project. This leads to complications when scanning medium to large environments.  

\subsection{Future solutions}
The modular structure provided by ROS in the autumn project allows it to be easily implemented using different hardware. For future projects it is possible to realize fully autonomous flight, using a smaller drone and much simpler two-dimensional lidar. This results in the loss of three-dimensional mapping capabilities but enables a more reliable and easier environment exploration.\newline 
Furthermore through using ROS, every part of the autumn logic can be used in separate projects where needed.
An example would be using the two-dimensional path-planning algorithm in order to plan ahead the movement of a ground vehicle. 

\filbreak