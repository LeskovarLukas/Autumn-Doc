\chapter{Conclusion}

\textbf{Author: Fabian Kleinrad} 

\section{Autumn Result}

The objective of the autumn project being a fully autonomous drone to be able to create 3D scans of hard to get to environments. Additionally on basis of the technologies used it is possible to deploy the drone in areas lacking a GPS signal. With these general conditions there was also an effort to keep hardware cost minimal. This leads to the final design of the autumn drone. A stereo camera was chosen due to low cost compared to technologies used for commercial autonomous solutions. The drone used in the autumn project provides for a wide range of possible sensor configuration and a high payload capacity. In the current version of the autumn drone only semi-autonomous flight is supported. Due to the lack of testing possibilities and the risk that accompanies testing autonomous drones, flight capabilities were only assessed using human input.   

\section{Mapping}
With the focus of autumn being, to be able to create 3D models of environments, mapping is a crucial part in the whole project. Mapping was realized using a SLAM algorithm, that enabled the use of a stereo camera.
SLAM generates a point cloud representing the environment the drone explores. All mapping logic is computed on the drone using a NVIDIA-Jetson, which provides enough computational power to ensure a clean model. In order to transfer the point cloud data, the jetson hosts an access point over which data is sent using ROS. This enables the user to get a real-time view of the model and how it is being constructed. The resulting point-cloud can then be used as reference material for example in a life-like render.

\section{Path-Planning}
Autonomous implies means to be able to move through an environment. In autumn this is realized via path-planning algorithm based on the model generated by the SLAM algorithm. The path is calculated separate from the drone and then sent over a wifi connection using ROS. By separating the path computation onto an external device it enables the drone to work more efficiently and the path-planning algorithm to not be constricted by computational restriction present when computing on the drone itself. The path is planned between the position of the drone and an user defined point. It gets checked each time new mapping data is available and adjusted if necessary. Depending on the mapping output path-planning can happen in two-dimensional or three-dimensional space. 

\section{Outlook}
At current times fully autonomous flight is not possible with the autumn drone. This is among other things due to the lack of sensors equipped to the drone. In the current stage the drone has no way to tell if obstacles are directly above it. This results in problems when using three dimensional path-planning because of the uncertainty of what lies above. Another problem is the short battery life-time of the drone, which is a result of the size of the drone.\newline
The modular structure provided by ROS in the autumn project allows it to be easily implemented using different hardware. For future projects it is possible to realize fully autonomous flight, using a smaller drone and a simple two-dimensional lidar. This results in the loss of three-dimensional mapping capabilities but enables more reliable and easier environment exploration. Furthermore through using ROS every part of the autumn logic can be used in separate projects where needed. An example would be using the two-dimensional path-planning algorithm in order to plan ahead the movement of an ground vehicle. 

\filbreak