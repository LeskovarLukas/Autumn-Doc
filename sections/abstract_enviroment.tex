\chapter{Representation of Environments}
\label{chapter:abstract_env}

\textbf{Author: Fabian Kleinrad} 

This chapter will concern itself with the critical intersection of physical and digital space. Fundamental for autonomous navigation are means for the algorithm to observe its surroundings. In order to accomplish a translation from physical space to a medium tailored towards information propagation, a SLAM algorithm, which is described in detail in chapter \ref{chapter:slam}, is being used. The following sections are going to explore the data, which is the output of the mapping phase and input into the path planning phase.

\section{Abstract Environments}

Abstraction being the process of reducing an Object to the information needed for further processing. In conjunction with autonomous navigation, this principle is ubiquitous. Due to computational space and time restrictions, a simplification of complex compounds is needed. Such a loss of information naturally happens when using equivalent hardware to process physical environments.\newline
For autonomous navigation to work, it is necessary for the computer to know the existence and position of objects surrounding the navigated vessel. This information can be provided or self-taught in case of using a SLAM algorithm. SLAM reduces the environment to positions in a two or 3-dimensional space. The algorithm is then left with an assortment of positions relative to the starting point, represented by the coordinate origin. To make use of this information, it is imperative to know the robot's position in this abstracted space.\newline
This approach of letting the algorithm perceive the environment guarantees an efficient and cost-effective computation. Moreover, by reducing obstacles to only the properties needed, unnecessary complexity is circumvented.      
The Kalman Filter described in section \ref{sec:ExtendedKalmanFilter}, reflects this principle.

\section{Representation in Autumn}
To utilize the information provided, there needs to be the capability of bundling data into a tried and tested format that enables easy access and reliable accuracy.\newline
Autumn uses the occupancy grid and point cloud to represent the data acquired in the mapping stage. The occupancy grid serves its purpose in a 2-dimensional environment, whereas on the contrary, the point cloud is utilized in 3-dimensional use cases.

\subsection{Occupancy Grid}
An occupancy grid is used for a 2-dimensional representation of the environment. In the case of an occupancy grid, a grid structure is being overlaid over the environment. Therefore it is possible to discretize the world into cells. Each cell contains information about the probability of being occupied. In this context, the assumption is made that cells can have either one of the two states. Thus, it simplifies the algorithms needed to update such data structures because this assumption allows for the use of binary random variables. Grid cells that are for certain occupied are represented by 1, which relates to the 100 per cent probability of being occupied. Contrary free cells are described by a 0 per cent probability of being occupied. Visually these features are displayed by colouring the cells black and white and every probability in between.\footcite{uni-freiburgOccupancyGridMaps2020}

\begin{figure}[h]
	\centering
	\includesvg[width=0.5\linewidth]{img/svg/OccupancyGridCells}
	\caption{Visual depiction of probability values in an occupancy grid map.}
	\label{fig:abstract_environments_occupancyCells}
\end{figure}

Two additional assumptions are made to add to the simplicity of the data structure and lower the time needed to update. First, the occupancy grid and many other data structures containing real-world information assume a static world, which entails that occupied cells and unoccupied cells will not change their value of occupation. Secondly, cells are viewed as individuals with no influence on their neighbouring cells. As a result, the calculation of probabilities is simplified. The probability is, in this case, the product of the probability of single sensor readings.\footcite{uni-freiburgOccupancyGridMaps2020}

\subsection{Voxel Grids}

When familiar with the concept of an occupancy grid, it would appear that adding a dimension would solve the problem of 3-dimensional representation.
Using an occupancy grid in such a use case would be called a voxel grid. The environment is not subdivided into two-dimensional planes but cubes with voxel grids. The basic principle of the occupancy grid stays the same. However, adding this additional dimension accentuates a drawback that accompanies the concept of occupancy grids. When working with occupancy grids, the space needed to store this kind of structure grows linearly with the number of cells. Using it in a 3-dimensional context makes larger scans hard to maintain and inefficient to work with, particularly in a real-time navigation scenario.\footcite{uni-freiburgOccupancyGridMaps2020}

\subsection{Point Cloud}
The most widespread data structure for storing 3-dimensional environments is the point cloud. In contrast to its adversaries, it stores only information that is known. The drawback of the voxel grid is thereby minimized. Point Clouds store points in three-dimensional space. These points represent an estimation of a point on a surface of an object, derived from sensor readings and the position these readings took place. Therefore, a Point Cloud is an accumulation of various points that represent the boundaries in which collision-free navigation can occur.
Additionally, the positional information stored in these points can also contain colour and luminescence values.\footcite{tech27PointCloud2018}\newline
Discarding a strict grid structure, which decreases the spacial requirements, also increases the complexity of working with this kind of data structure. In robotics, almost always are point clouds used in conjunction with the point cloud library.

\subsubsection{PCL}
The Point Cloud Library simplifies working with a point cloud data structure. PCL is a free library supporting point clouds of any dimension. It is written for optimal performance and offers a variety of 3D processing functionality. These include filtering, segmentation, feature estimation. Furthermore, PCL is fully integrated into ROS, which means specific functionalities are packaged into nodelets that can be added to fit the underlying use case.\footcite{ConselhoNacionaldeDesenvolvimentoCientificoeTecnologico1995}

\section{Abstracted Environments in Autumn}
Autumn contains two different path planning packages catering to different dimensional environments. When using a drone, three-dimensional path planning is necessary to benefit from the usage of an aerial vehicle. Therefore the "autumn\_pathfinding" is used, based on a point cloud. Additionally, the package "autumn\_pathfinding\_2D" uses an occupancy grid and thus is appropriate for situations where the third dimension cannot be utilized.\newline
Through the use of ROS, explained in detail in Chapter \ref{chapter:ros}, this modular approach is made possible. Furthermore, the Point Cloud and Occupancy Grid data types are integrated into ROS and universally applicable.   

\begin{figure}[h]
	\centering
	\includesvg[width=0.8\linewidth]{img/svg/EnvironmentTransfer}
	\caption{Excerpt of the Autumn-Life-Cycle. Lifespan of point cloud and occupancy grid. Created in the SLAM stage, published as a topic and used in the navigation stage.}
	\label{fig:abstract_environments_enviromentTransfer}
\end{figure}

\subsection{autumn\_pathfinding\_2D}
The package "autumn\_pathfinding\_2D" subscribes to a topic published from the slam node, which carries a nav\_msgs/OccupancyGrid. This is a standardized data type for occupancy maps. It consists of a header of type std\_msgs/Header, containing a sequence number, timestamp, and frame id used to identify individual messages. Additionally, the map's metadata and the current data contained in an int8 array are transferred over this topic.\footcite{rosNavMsgsOccupancyGrid2021}\footcite{rosStdMsgsHeader2021}\newline
Upon receiving a new map and goal point, given that current calculations are ongoing, the algorithm aligns a Cartesian coordinate system relative to the drone's current position. From this point onward, access to the occupancy grid represented by the int8 array happens through a function that takes x and y coordinates and maps it to the one-dimensional array. This is essential to guarantee usability alongside performance.\newline
The coordinates get also adjusted based on the occupancy grid resolution. Benefits gained from such a procedure are that it is easy to check duplicate cells because only whole cells can be represented. Another reason for it is the use of the cantor pairing function\footcite{Szudzik2017} which only works with natural numbers. The reason for using such a pairing function is to store and search for points efficiently.  
The underlying path planning algorithm explained in detail in chapter \ref{chapter:path_planning} is not affected by the input data structure.       
The data array, holding occupancy information, contains per index a value between 0 and 100 or, if the cell is unknown, the value -1. The path planning algorithm works in a way that unknown cells with value -1 are taken as free cells. This "free until proven occupied" mentality allows for more efficient space exploration.

\subsection{autumn\_pathfinding}
The three dimensional equivalent to the two-dimensional path planning packages uses a point cloud in order to navigate the environment. This point cloud is published as a sensor\_msgs/PointCloud message type. This universal point cloud type contains a header with message-specific information as well as an array of geometry\_msgs/Point32, which represent the individual points of the point cloud. In addition, Point32 contains x, y and z attributes representing the Cartesian coordinates.\footcite{rosGeometryMsgsPoint322021}\footcite{rosSensorMsgsPointCloud2021}\newline
The point cloud library will be used because of the difficulty of working with a three-dimensional space represented by a one-dimensional array. Same as the two-dimensional model, the algorithm of the three-dimensional path planning accesses the point cloud data through a function that takes three coordinates and returns if an object exists at this position.\newline
Like the 2d solution, point coordinates get adjusted to a specific resolution. However, in this case, it is not bound to the data type itself because of the lack of structure in a point cloud. Instead, the point resolution is chosen to guarantee the best performance while retaining a high level of flexibility. This process of adjusting the resolution can be understood as altering the size of cubes covering the space, like cells in an occupancy grid. This way, a structure is added to the otherwise unstructured set of points, which allows for the exact implementation of the algorithm as in the two-dimensional solution. 


