\chapter{Representation of Environments}

\textbf{Author: Fabian Kleinrad} 

This chapter is going to concern itself with the critical intersection of physical and digital space. Fundamental for autonomous navigation to work are means for the algorithm to observe its surrounding. In order to accomplish a translation from physical space to a medium tailored towards information propagation, a SLAM algorithm, which is described in detail in \ref{chapter:slam}, is being used. These following sections are going to explore the data, which is the output of the mapping phase and input into the path planning phase.

\section{Abstract Environments}

\["The\ situation\ in\ which\ a\ subject\ is\ very\ general \ and\ not\ based\ on\ real\ situations"\]
Abstraction being the process of reducing an Object to the information needed for further processing. 

\subsection{Abstraction Levels}

\section{Abstraction Methods in Autumn}

\subsection{Occupancy Grid}

\subsection{Point Cloud}

\subsection{ROS Types}

\section{Abstracted Environments in Autumn}

\subsection{autumn$\_$pathfinding$\_$2D}

\subsection{autumn$\_$pathfinding}